\section{Results}
\subsection{K-Nearest Neighbors}

The first of our 3 methods performed quite poorly. Table \ref{fig:knn} shows the test accuracy for our keypoint extracted nearest neighbor classifiers. We tried K values ranging from 1 to 100 but found as expected that the elbow point for this dataset was quite low. We concluded that the best value was actually 1 neighbor.
\begin{table}[h!]
\centering
\begin{tabular}{|c|c|}\hline
	\textbf{K value} & \textbf{Test Accuracy}\\ \hline
	 1 & 0.0965\\ \hline
	 2  & 0.084\\ \hline
	5  & 0.0733\\ \hline
	10  & 0.04\\ \hline
\end{tabular}
\caption{\label{fig:knn} Shows the accuracy for various values of K in the ORB-Nearest Neighbors}
\end{table}

\subsection{Transfer learning}

The table below highlights our performance with some specified values of the type of model, number of layers used, number of epochs. These results were obtained with a learning rate $lr = 0.01$.

\begin{table}[h!]
	\centering
	\begin{tabular}{|c|c|c|c|c|c|}\hline
		\textbf{Model} & \textbf{Layers} & \textbf{\# epochs} & \textbf{Train accuracy} & \textbf{Val accuracy} & \textbf{Test accuracy}\\ \hline
		ResNet18  & Last only & 50 & 54.33\% & 24\% & 3.61\%\\ \hline
		ResNet18  & All & 20 & 81.16\% & 58\% & 8.491\%\\ \hline
		ResNet152 & All & 15 & 84.19\% & 55\% & -\\ \hline
	\end{tabular}
	\caption{\label{tab:transferresults}Results for transfer learning with various techniques}
\end{table}

We tried lower learning rates in the range [0.001, 0.01], but these learned too slowly to produce good results. For example, running ResNet18 with learning rate $lr = 0.001$ and training only the fully-connected layer, after 200+ epochs our model only got to around 50\% accuracy and 32\% validation accuracy. That being said, we think that lower learning rates do overall produce better results. Unforunately, we were unable to record exact results of a lot of earlier runs with fewer preprocessing steps due to misconfigurations and time constraints. One of the higher test scores that we verifiably achieved was 15.716\% on the public leaderboard, with an objectively weaker preprocessing pipeline. Also, further training of this same model reduced the accuracy, presumably because of overfitting.

\subsection{Bounding Boxes and ResNet50}

It was challenging for us to correctly apply homographies computed. In large part this might be because there was too much noise in the data, and the computed homographies were not very good ones. Here our model learned much more quickly, but it became tough for us to reduce overfitting. Partially freezing some of the layers did not work as expected. However, despite all this we recieved a modest validation bump of about 3-5\% on average from the Transfer learning models. Our test accuracy however, was not too much higher at 16\%. This probably had to do with tuning hyperparameters. This approach requires a little more understanding on our part. 

